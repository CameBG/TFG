%%%%%%%%%%%%%%%%%%%%%%%%%%%%%%%%%%%%%%%%%%%%%%%%%%%%%%%%%%%%%%%%%%%%%%%%
% Plantilla TFG/TFM
% Escuela Politécnica Superior de la Universidad de Alicante
% Realizado por: Jose Manuel Requena Plens
% Contacto: info@jmrplens.com / Telegram:@jmrplens
%%%%%%%%%%%%%%%%%%%%%%%%%%%%%%%%%%%%%%%%%%%%%%%%%%%%%%%%%%%%%%%%%%%%%%%%

\chapter{Introducción}

\section{Motivación y contexto}



\\
	Mi interés por la visión artificial se desarrolló durante el tercer año de carrera, donde me presenté junto a la Universidad de Alicante a un proyecto llamado Vodafone Campus Lab[cita??], donde te proponen unos problemas a resolver y nosotros decidimos plantear una solución con tecnologías como la visión artificial, además esta propuesta requería de procesos como la adquisición del cuerpo 3D, reconocimiento de este, entre otras. Aquí fue cuando busqué propuestas de trabajo similares o sobre este tema, porque me quedé con la curiosidad de llevarlo a cabo y porque quería saber y aprender más sobre este tema, y conocí el proyecto de investigación Tech4Diet [cita]
\\
\\
	El proyecto de investigación Tech4Diet cuenta con el apoyo de la Agencia Estatal de Investigación (AEI) y del Fondo Europeo de Desarrollo Regional (FEDER) con referencia "TIN2017-89069-R" perteneciente al programa Retos 2017 en el que su investigador jefe es Jorge Azorín. En este proyecto se busca facilitar el estudio de la evolución morfológica ocasionada por tratamientos de obesidad. Hoy en día, estos tratamientos son muy costosos pero a su vez muy necesarios, ya que los problemas de obesidad o sobrepeso pueden ocasionar enfermedades crónicas como la hipertensión, diabetes tipo II, cáncer. También pueden ocasionar enfermedades patológicas neurodegenerativas como el Alzheimer o demencias [Fuster-Guilló et al., 2020]
\\
\\
	El sistema utilizado para esta finalidad, dispone de una red de cámaras RGB-D que
	obtienen un modelo 3D del cuerpo del paciente. El proceso de obtención de un modelo 3D se realiza en diferentes sesiones médicas, lo que permite una visualización real de la
	evolución del cuerpo del paciente. Para que el paciente pueda visualizar su progreso, no solo dispondrá de una aplicación de ordenador, sino que también podrá visualizar los
	diferentes modelos de su cuerpo mediante unas gafas de realidad virtual. La realidad virtual tiene como finalidad incrementar la adherencia del usuario al tratamiento. Además, podemos encontrar desarrollos tecnológicos como "Google Cardboard" que nos permiten convertir cualquier teléfono móvil en unas gafas de realidad virtual sin necesidad de realizar un gasto de dinero elevado. A parte de la visualización, sobre este modelo 3D obtenido con las cámaras se pueden realizar mediciones de diferentes partes del cuerpo a niveles de 1D, 2D y 3D.
\\

Para que todo esto sea efectivo las mediciones han de corresponderse a las reales, por lo tanto el objetivo principal de este trabajo es comprobar y calcular si se pueden obtener buenas mediciones a partir de un modelo 3D generado con una sola imagen del cuerpo de la persona que será procesado en una red utilizando así el proyecto PIFu[cita] para ello.



\section{Estado del arte}



%\begin{verbatim}
%\bibliographystyle{apacite}
%\end{verbatim}


%\item[Anexos:] se podrán incluir los anexos que se consideren oportunos.



\section{Objetivos}


%\par Esto es una cita estándar: \citet{Shaw1996}, que también puedes mostrar con paréntesis así: \citep{Shaw1996}. También se puede realizar una cita indicando a qué parte te refieres \citep[ver][Cap. 2]{Shaw1996} o \citep[Cap. 2]{Shaw1996} o \citep[ver][]{Shaw1996}. 

%\par También puedes mostrar todos los autores cuando hay más de 2 autores añadiendo un asterisco después del comando como: \citet*{Akyildiz2005}, sin el asterisco quedaría así: \citet{Akyildiz2005}.

%\par O puedes citar dos o más fuentes al mismo tiempo: \citep{Barkan1995,Leighton2012}

Este trabajo esta enmarcado dentro de un proyecto de investigación, y por lo tanto comparte el mismo objetivo general, proporcionar un sistema capaz de estudiar la evolución del cuerpo humano con el paso del tiempo mediante técnicas de visión 3D. 

Para llegar al objetivo general es necesario disponer de una buena obtención del modelo 3D. Para ello, tenemos los siguientes objetivos:


\begin{itemize}
	\item{\textbf{Objetivo 1: Obtención del modelo 3D a partir de una imagen usando PIFu como red.}} 
	\\Para ello se realizarán tareas como:
	\begin{itemize}
		\item Estudio de los diferentes parámetros necesarios para poder realizar dicho proceso.
		\item Generación de máscara de la imagen.
		\item Estudio de los resultados obtenidos.
		
		
	\end{itemize}
	\item{\textbf{Objetivo 2: Comparativa del modelo 3D obtenido con el modelo obtenido por el proyecto Tech4Diet. 
			.}} \\
	Para este objetivo se tendrán que realizar las siguientes tareas específicas:
	\begin{itemize}
		\item Estudio sobre las diferentes formas de realizar el cálculo de las diferencias obtenidas.
		\item Calculo de distancias usando Hausdorff Distance
		\item Calculo de distancias usando 
	\end{itemize}
\end{itemize}

\label{sec:Introducción}

