%%%%%%%%%%%%%%%%%%%%%%%%%%%%%%%%%%%%%%%%%%%%%%%%%%%%%%%%%%%%%%%%%%%%%%%%
% Plantilla TFG/TFM
% Escuela Politécnica Superior de la Universidad de Alicante
% Realizado por: Jose Manuel Requena Plens
% Contacto: info@jmrplens.com / Telegram:@jmrplens
%%%%%%%%%%%%%%%%%%%%%%%%%%%%%%%%%%%%%%%%%%%%%%%%%%%%%%%%%%%%%%%%%%%%%%%%

\chapter*{Resumen}
\thispagestyle{empty}


Actualmente hay muchos estudios y proyectos llevados a cabo para la reconstrucción del cuerpo humano 3D, con enfoques tanto como hacia el deporte, los videojuegos, o más científicos como el sanitario o nutricional.   

Este trabajo está dentro de un marco dietético nutricional de la obesidad, que tiene como objetivo, buscar una solución que represente el cuerpo humano 3D, con el uso de imágenes obtenidas por cámaras de consumo. En concreto, se han utilizado las cámaras disponibles en los teléfonos móviles. 

Dado que estamos en un ámbito sanitario, necesitamos que las medidas sean correctas al mayor porcentaje posible. Para comprobarlo se va realizar una comparación con otro proyecto llamado Tech4diet [\cite{tech}], que es capaz de representar la forma y textura de forma precisa dado que realiza esta adquisición con una red de cámaras RGBD. 

Para abordar el objetivo se propone el uso de métodos basados en aprendizaje profundos capaces de proporcionar la reconstrucción 3D del cuerpo a partir de una sola vista 2D de la persona.


\cleardoublepage %salta a nueva página impar
\chapter*{Agradecimientos
}

\thispagestyle{empty}
\vspace{1cm}

Este trabajo no habría sido posible sin el apoyo de mis tutores Andrés y Jorge, como de Nahuel, su paciencia, disposición y comprensión tanto con mis dudas como conmigo misma, han sido clave durante este proyecto.

También quiero agradecer a mis amigos, que por muy lejos que estén han estado cerca de mí apoyándome y animándome para terminar mis estudios en uno de los momentos más complejos de mi vida.

Agradecer a mi familia, que siempre han tratado de darme soluciones a mis problemas apoyarme y animarme para terminar este ciclo.

No puedo terminar estos agradecimientos sin mencionar a mis compañeros de la Universidad de Alicante; Luis, Sandra, Edgar, Dani y el resto de personas que han estado a mi lado, gracias por los años que he estado a vuestro lado y por toda la ayuda y apoyo que nos hemos brindado, estoy muy agradecida de haberos tenido a mi lado durante mis años de carrera.

Es a ellos a quien dedico este trabajo.


\cleardoublepage %salta a nueva página impar
% Aquí va la dedicatoria si la hubiese. Si no, comentar la(s) linea(s) siguientes
\chapter*{}
\setlength{\leftmargin}{0.5\textwidth}
\setlength{\parsep}{0cm}
\addtolength{\topsep}{0.5cm}
\begin{flushright}
\small\em{
A mi abuela Carmen, que no ha podido verme terminar esta etapa.\\
}
\end{flushright}

\cleardoublepage %salta a nueva página impar
% Aquí va la cita célebre si la hubiese. Si no, comentar la(s) linea(s) siguientes
%\chapter*{}
%\setlength{\leftmargin}{0.5\textwidth}
%\setlength{\parsep}{0cm}
%\addtolength{\topsep}{0.5cm}
%\begin{flushright}
%\small\em{
%Si consigo ver más lejos\\
%es porque he conseguido auparme\\ 
%a hombros de gigantes
%}
%\end{flushright}
%\begin{flushright}
%\small{
%Isaac Newton.
%}
%\end{flushright}
%\cleardoublepage %salta a nueva página impar
