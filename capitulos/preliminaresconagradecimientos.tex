%%%%%%%%%%%%%%%%%%%%%%%%%%%%%%%%%%%%%%%%%%%%%%%%%%%%%%%%%%%%%%%%%%%%%%%%
% Plantilla TFG/TFM
% Escuela Politécnica Superior de la Universidad de Alicante
% Realizado por: Jose Manuel Requena Plens
% Contacto: info@jmrplens.com / Telegram:@jmrplens
%%%%%%%%%%%%%%%%%%%%%%%%%%%%%%%%%%%%%%%%%%%%%%%%%%%%%%%%%%%%%%%%%%%%%%%%

\chapter*{Resumen}
\thispagestyle{empty}

El aprendizaje profundo es un enfoque de aprendizaje automático no supervisado (es decir, son necesarios datos de entrenamientos, pero estos no requieren ser etiquetados) que se asemeja o se fundamenta en el funcionamiento del sistema neurológico humano.

La razón principal que me ha llevado a la realización de este trabajo ha sido el uso del aprendizaje profundo, dado que está cada vez más en auge además de la propia curiosidad sobre el aprendizaje y funcionamiento de este.

Este trabajo tiene como finalidad el estudio del aprendizaje profundo dentro de un marco dietético nutricional de la obesidad aprovechando las tecnologías para abordar este problema proponiendo una solución de la representación del cuerpo humano 3D con el uso único de imágenes y realizar unas comparativas sobre los modelos conseguidos para comprobar la veracidad de los datos obtenidos a partir de una red neuronal.



\cleardoublepage %salta a nueva página impar
\chapter*{Agradecimientos
}

\thispagestyle{empty}
\vspace{1cm}

Este trabajo no habría sido posible sin el apoyo de mis tutores, como de Nahuel, su paciencia, disposición y comprensión tanto con mis dudas como conmigo misma, han sido clave durante este proyecto.

También quiero agradecer a mis amigos, que por muy lejos que estén han estado cerca de mí apoyándome y animándome para terminar mis estudios en uno de los momentos más complejos de mi vida.

Agradecer a mi familia, que siempre han tratado de darme soluciones a mis problemas apoyarme y animarme para terminar este ciclo.

No puedo terminar estos agradecimientos sin mencionar a mis compañeros de la Universidad de Alicante; Luis, Sandra, Edgar, Dani y el resto de personas que han estado a mi lado, gracias por los años que he estado a vuestro lado y por toda la ayuda y apoyo que nos hemos brindado, estoy muy agradecida de haberos tenido a mi lado durante mis años de carrera.

Es a ellos a quien dedico este trabajo.


\cleardoublepage %salta a nueva página impar
% Aquí va la dedicatoria si la hubiese. Si no, comentar la(s) linea(s) siguientes
\chapter*{}
\setlength{\leftmargin}{0.5\textwidth}
\setlength{\parsep}{0cm}
\addtolength{\topsep}{0.5cm}
\begin{flushright}
\small\em{
A mi abuela Carmen, que no ha podido verme terminar esta etapa.\\
}
\end{flushright}

\cleardoublepage %salta a nueva página impar
% Aquí va la cita célebre si la hubiese. Si no, comentar la(s) linea(s) siguientes
%\chapter*{}
%\setlength{\leftmargin}{0.5\textwidth}
%\setlength{\parsep}{0cm}
%\addtolength{\topsep}{0.5cm}
%\begin{flushright}
%\small\em{
%Si consigo ver más lejos\\
%es porque he conseguido auparme\\ 
%a hombros de gigantes
%}
%\end{flushright}
%\begin{flushright}
%\small{
%Isaac Newton.
%}
%\end{flushright}
%\cleardoublepage %salta a nueva página impar
