%%%%%%%%%%%%%%%%%%%%%%%%%%%%%%%%%%%%%%%%%%%%%%%%%%%%%%%%%%%%%%%%%%%%%%%%
% Plantilla TFG/TFM
% Escuela Politécnica Superior de la Universidad de Alicante
% Realizado por: Jose Manuel Requena Plens
% Contacto: info@jmrplens.com / Telegram:@jmrplens
%%%%%%%%%%%%%%%%%%%%%%%%%%%%%%%%%%%%%%%%%%%%%%%%%%%%%%%%%%%%%%%%%%%%%%%%

\chapter{Conclusiones}
\label{conclusiones}
En este capítulo se resaltan las principales conclusiones obtenidas como consecuencia del trabajo desarrollado y a su vez líneas futuras derivadas al trabajo.

\section{Conclusión}

En el presente trabajo se han planteado una serie de objetivos relacionados con la necesidad de poder ser capaces de generar un modelo del cuerpo humano lo más real posible con el uso de una cámara de teléfono móvil y con una sola vista. 

Se ha seleccionado el uso de una cámara de teléfono móvil porque actualmente toda persona adulta tiene uno, y por lo tanto es un sistema muy flexible y por supuesto portable.

En relación al estudio de la red profunda que se ha usado, hemos podido comprobar que ha sido alimentada y entrenada por modelos que tienen un físico más normativo, dado que cada vez que se ha probado con un cuerpo menos normativo más errores nos hemos encontrado en el proceso, como por ejemplo una "pérdida" de peso considerable en comparación cuando observabas los modelos en ángulos más ladeados.

En relación al estudio de los resultados obtenidos se concluye con que hay ciertos ángulos que favorecen a la red, hemos podido comprobar que el ángulo donde se ve la espalda nos ha proporcionado modelos con menos problemas, pero que este estudio es mejorable. Se ha repetido mucho que la vista se ve perfecta desde el ángulo de la imagen pero cuando empiezas a mirar el resto de ángulos se ven muchos desperfectos y deformidades.


\section{Líneas Futuras}

Como consecuencia del trabajo desarrollado se han derivado una serie de líneas futuras entre las que podemos destacar:

Ampliación del uso de la red a multivista, es decir pasar de una vista a varias, ya que como se ha comentado en las conclusiones, se puede observar que usualmente el ángulo utilizado como imagen, se observa que lo representa bien, el problema ocurre una vez se observa desde más ángulos, esto se soluciona con diferentes vistas a la vez, y esta red se puede exportar a ello.

Entrenamiento con diferentes modelos sobre diferentes tipos de cuerpos humanos para obtener mejores resultados, y prepararlo para multivista.

Comparación de los resultados obtenidos con multivista y una vista.
